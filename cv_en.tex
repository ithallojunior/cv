\documentclass[11pt]{article}
\usepackage[utf8]{inputenc}
\usepackage[margin=1.0in]{geometry}
\usepackage[colorlinks=true, 
            urlcolor=blue,
            pdfauthor={Ithallo Junior Alves Guimaraes},
            pdftitle={CV de Ithallo},
            ]{hyperref}
\usepackage{scrextend} %addmargin
\usepackage{multicol} % multiline columns for itemize
\setlength{\parindent}{0cm}

\begin{document}

\begin{center}
\huge{Ithallo Junior Alves Guimarães}
\end{center}

\section{Personal Information}

\begin{addmargin}{0.5cm}

\textbf{Address:}  QN 7 AE 7, Riacho Fundo 1, 71805-737, Brasília - DF. \\
\textbf{Telephone:}  +55 61 99951-7324 \\
\textbf{E-mail:}   \href{maito:ithallojunior@outlook.com}{ithallojunior@outlook.com} \\
\textbf{Website:}  \url{https://ithallojunior.github.io} or \url{http://ithallo.ml} \\
\textbf{GitHub:} \url{https://github.com/ithallojunior}

\end{addmargin}


\section{Formal Education:}

\textbf{2018 - present day}
\begin{addmargin}{0.5cm}
Master's in Biomedical Engineering - University of Brasília (UnB), Brasília, Brazil. \\
\end{addmargin}

\textbf{2012 - 2017}
\begin{addmargin}{0.5cm}
Undegraduated in Electronics Engineering - University of Brasília (UnB), Brasília, Brazil. \\
% Tema da monografia: Desenvolvimento tecnológico de um dispositivo de coleta
%de sinais sEMG aplicado à avaliação da doença de \textit{Parkinson}.\\
\end{addmargin}

\textbf{2014 - 2015} 
\begin{addmargin}{0.5cm}
Exchange Student (Electrical Engineering) - University of Technology and Economics of Budapest (BME), Budapest, Hungary. 
\end{addmargin}

\section{Experiência:}

\textbf{março 2018 - presente}
\begin{addmargin}{0.5cm}
Scholarship Holder - Coordenadoria de Aperfeiçoamento de Pessoal de Nível Superior (CAPES). \\
Bolsista pelo Programa de Pós-Graduação em Engenharia Biomédica
da Faculdade Gama da University of Brasília (PPGEB FGA-UnB).\\
Linha de pesquisa: Análise e desenvolvimento de sistemas inteligentes e de saúde.\\
\end{addmargin}

\textbf{outubro 2018 - dezembro 2018}
\begin{addmargin}{0.5cm}
Voluntary Professor -  Gama Faculty, University of Brasília (FGA-UnB).\\
Professor de Prática de Eletrônica Digital 1. \\
\end{addmargin}

\textbf{março 2018 - julho 2018}
\begin{addmargin}{0.5cm}
Estágio em docência - Faculdade Gama, University of Brasília (FGA-UnB).\\
Aulas ministradas na disciplina de Inteligência Artificial sobre Redes 
Neurais Artificiais e Algoritmos Genéticos. \\
\end{addmargin}

\textbf{setembro 2017 - fevereiro 2018}
\begin{addmargin}{0.5cm}
Estagiário - LoopKey.\\
Desenvolvimento de sistemas de comunicação, automação e controle via Bluetooth
em sistemas Linux  embarcados utilizando-se Python. \\
\end{addmargin}

\newpage

\textbf{janeiro 2016 - janeiro 2017}
\begin{addmargin}{0.5cm}
Bolsista - Fundação de Apoio a Pesquisa do Distrito Federal (FAP-DF).\\
Bolsista de Iniciação Científica no Laboratório de Informática em Saúde (LIS)  da Universidade de
Brasília (UnB) no projeto
de desenvolvimento tecnológico para adaptação de membros artificiais em amputados transfemorais,
trabalhando com desenvolvimento de circuitos eletrônicos, Python e aplicações de modelos de 
\textit{Machine Learning}.\\

\end{addmargin}

    \textbf{setembro 2014 - agosto 2015}
    \begin{addmargin}{0.5cm}
        Bolsista - Coordenadoria de Aperfeiçoamento de Pessoal de Nível Superior (CAPES). \\
        Bolsista de intercâmbio pelo programa Ciência sem Fronteiras na Universidadede Tecnologia e Economia de Budapeste (BME). 
\end{addmargin}

\section{Habilidades e conhecimentos:}

\subsection{Idiomas:}
\begin{center}
\begin{tabular}{c | c | c | c}
\hline
\textbf{Língua:} & Fala & Leitura & Escrita \\
\hline
\hline
\textbf{Inglês} &  Avançado & Avançado & Avançado  \\
\hline
\textbf{Espanhol} & Básico & Básico & Básico \\
\hline
\textbf{Francês} & Básico & Básico & Básico \\
\hline
\end{tabular}
\end{center}

\subsection{\textit{Softwares}, sistemas, linguagens de programação e outras habilidades:}
\begin{multicols}{3} %number of columns
\begin{itemize}
    \item Python;
    \item C;
    \item Shell;
    \item C\#;
    \item \textit{Machine Learnig};
    \item \LaTeX ;
    \item Git;
    \item AngularJS;
    \item Matlab;
    \item Octave;
    \item Julia;
    \item Office (Word, Excel e PowerPoint);
    \item LibreOffice;
    \item Pages, Numbers e Keynote;
    \item Windows, GNU/Linux e MacOS (OS X);
    \item Solda;
    \item Captura de sinais biomédicos;
    \item Filtros analógicos e digitais;
    \item Arduíno;
    \item Processamento digital de sinais.
\end{itemize}
\end{multicols}


\end{document}
