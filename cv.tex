\documentclass[11pt]{article}
\usepackage[utf8]{inputenc}
\usepackage[T1]{fontenc} %corrects accents
\usepackage[margin=1.0in]{geometry}
\usepackage[colorlinks=true, 
            urlcolor=blue,
            pdfauthor={Ithallo Junior Alves Guimaraes},
            pdftitle={CV de Ithallo},
            ]{hyperref}
\usepackage{scrextend} %addmargin
\usepackage{multicol} % multiline columns for itemize
\setlength{\parindent}{0cm}

\begin{document}

\begin{center}
\huge{Ithallo Junior Alves Guimarães}
\end{center}

\section{Informações pessoais:}
\hrule \vspace{0.1cm}
\begin{addmargin}{0.5cm}

\textbf{Residência}: Brasília, DF, Brasil \\
\textbf{Telefone:}  +55 61 99951-7324 \\
\textbf{Email:}   \href{maito:ithallojunior@outlook.com}{ithallojunior@outlook.com} \\
%\textbf{Site:}  \url{https://ithallojunior.github.io}  \\
\textbf{GitHub:} \url{https://github.com/ithallojunior} \\
\textbf{Linkedin:} \url{https://linkedin.com/in/ithallo/}
\end{addmargin}

\section{Apresentação}
\hrule \vspace{0.1cm}

\begin{addmargin}{0.5cm}

Desenvolvedor Python e Go/Golang e mestre em Engenharia Biomédica com formação em Engenharia Eletrônica. 
Experiência em Python, Go/Golang, Docker Kubernetes, \textit{Machine Learning}, instrumentação biomédica (e 
eletrônica), aquisição de sinais biomédicos, processamento digital de sinais,
entre outras expertises. Autodidata, determinado e 
capaz de aprender rapidamente. Possui Inglês fluente e Francês e Espanhol básicos.

\end{addmargin}

\section{Formação:}
\hrule \vspace{0.1cm}

\textbf{2018 - 2021}
\begin{addmargin}{0.5cm}
Mestrado em Engenharia Biomédica - Universidade de Brasília (UnB), Brasília - DF.\\ 
Tema da dissertação: Desenvolvimento de um eletromiógrafo de superfície sem fio e
comparação com o Delsys Bagnoli.\\
\end{addmargin}

\textbf{2012 - 2017}
\begin{addmargin}{0.5cm}
Graduação em Engenharia Eletrônica - Universidade de Brasília (UnB), Brasília - DF. \\
Tema da monografia: Desenvolvimento tecnológico de um dispositivo de coleta
de sinais sEMG aplicado à avaliação da doença de \textit{Parkinson}.\\
\end{addmargin}

\textbf{2014 - 2015} 
\begin{addmargin}{0.5cm}
Intercâmbio (Engenharia Elétrica) - Universidade de Tecnologia e Economia de Budapeste (BME), Budapeste, Hungria. 
\end{addmargin}

\section{Experiência:}
\hrule \vspace{0.1cm}

\textbf{setembro 2021 - presente}
\begin{addmargin}{0.5cm}
Analista de \textit{Software} Sênior - Instituto de Pesquisas Eldorado \\
Atuação em diferentes projetos:
\begin{itemize}
    \item Líder técnico e desenvolvedor. Tecnologias: Python (PyQt, Flask e SQLAlchemy), Microsoft SQL Server, SQLite, Docker e linux embarcado;
    \item Desenvolvedor. Tecnologias: Go/Golang (Operator-SDK), Docker, Kubernetes, Prometheus, Grafana e TimescaleDB. \\
\end{itemize}
\end{addmargin}

\newpage
\textbf{maio 2021 - setembro 2021}
\begin{addmargin}{0.5cm}
Analista de \textit{Software} - Instituto de Pesquisas Eldorado \\
Líder técnico e desenvolvedor. Tecnologias: Python (PyQt, Flask e SQLAlchemy), Microsoft SQL Server, SQLite, Docker e linux embarcado.\\
\end{addmargin}


\textbf{dezembro 2020 - maio 2021}
\begin{addmargin}{0.5cm}
    Analista de Sistemas e Desenvolvimento I - Coopersystem \\
    Desenvolvedor principal \textit{back-end} Python no projeto Portal Coopersystem,
    utilizando as seguintes tecnologias: Django, Django REST Framework, Celery, 
    RabbitMQ, PostgreSQL e Docker.\\
\end{addmargin}

\textbf{setembro 2019 - dezembro 2020}
\begin{addmargin}{0.5cm}
Programador II - Coopersystem \\
Desenvolvedor principal \textit{back-end} Python no projeto Portal Coopersystem,
utilizando as seguintes tecnologias: Django, Django REST Framework, Celery, 
RabbitMQ, PostgreSQL e Docker.\\
\end{addmargin}

\textbf{janeiro 2019 - dezembro 2021}
\begin{addmargin}{0.5cm}
Professor particular. \\
Professor particular, ministrando aulas de reforço em matérias diversas
do ensino superior (cálculo, eletrônica digital, programação, etc.), médio
e fundamental.\\
\end{addmargin}

\textbf{março 2018 - setembro 2019}
\begin{addmargin}{0.5cm}
Bolsista - Coordenadoria de Aperfeiçoamento de Pessoal de Nível Superior (CAPES). \\
Bolsista pelo Programa de Pós-Graduação em Engenharia Biomédica
da Faculdade Gama da Universidade de Brasília (PPGEB FGA-UnB).\\
Linha de pesquisa: Análise e desenvolvimento de sistemas inteligentes e de saúde.\\
\end{addmargin}

\textbf{outubro 2018 - dezembro 2018}
\begin{addmargin}{0.5cm}
Professor voluntário - Faculdade Gama, Universidade de Brasília (FGA-UnB).\\
Professor de Prática de Eletrônica Digital 1. \\
\end{addmargin}

\textbf{março 2018 - julho 2018}
\begin{addmargin}{0.5cm}
Estágio em docência - Faculdade Gama, Universidade de Brasília (FGA-UnB).\\
Aulas ministradas na disciplina de Inteligência Artificial sobre Redes 
Neurais Artificiais e Algoritmos Genéticos. \\
\end{addmargin}

\textbf{setembro 2017 - fevereiro 2018}
\begin{addmargin}{0.5cm}
Desenvolvedor Python - LoopKey.\\
Desenvolvimento de sistemas de comunicação, automação e controle via Bluetooth
(BTLE) em sistemas Linux embarcados com Python e bibliotecas como
Bluepy e Flask. \\
\end{addmargin}

\newpage
\textbf{janeiro 2017 - dezembro 2017}
\begin{addmargin}{0.5cm}
Voluntário - Laboratório de Informática em Saúde (LIS). \\
Voluntário no Laboratório de Informática em Saúde (LIS) da Universidade de
Brasília (UnB) no projeto
de desenvolvimento tecnológico para adaptação de membros artificiais em amputados transfemorais,
trabalhando com desenvolvimento de circuitos eletrônicos, Python, aplicação de modelos de 
\textit{Machine Learning}, análise e coletas de sinais em pacientes.\\
\end{addmargin}

\textbf{janeiro 2016 - janeiro 2017}
\begin{addmargin}{0.5cm}
Bolsista - Fundação de Apoio a Pesquisa do Distrito Federal (FAP-DF).\\
Bolsista de Iniciação Científica no Laboratório de Informática em Saúde (LIS)  da Universidade de
Brasília (UnB) no projeto
de desenvolvimento tecnológico para adaptação de membros artificiais em amputados transfemorais,
trabalhando com desenvolvimento de circuitos eletrônicos, Python, aplicação de modelos de 
\textit{Machine Learning}, análise e coletas de sinais em pacientes.\\
\end{addmargin}

\textbf{setembro 2014 - agosto 2015}
    \begin{addmargin}{0.5cm}
        Bolsista - Coordenadoria de Aperfeiçoamento de Pessoal de Nível Superior (CAPES). \\
        Bolsista de intercâmbio pelo programa Ciência sem Fronteiras na Universidade de Tecnologia e Economia de Budapeste (BME). 
\end{addmargin}


\section{Publicações:}
\hrule \vspace{0.1cm}

\textbf{Human Knee Simulation using Multilayer Perceptron Artificial Neural Network.}
Guimarães, I.J.A.; Lima, R.A.; Marães, V.R.F.S.; Brasil, L.M.
XXV Congresso Brasileiro de Engenharia Biomédica – CBEB. 2016. \\

\textbf{Classificação de padrões de marcha utilizando-se de diferentes algoritmos de aprendizado de máquinas.}
Junior, J.L.F.S; Sousa, B.S; Guimarães, I.J.A.; Marães, V.R.F.S.; Brasil, L.M.
Anais do V Congresso Brasileiro de Eletromiografia e Cinesiologia e X Simpósio de Engenharia Biomédica. 2017 \\

\textbf{CLASSIFICAÇÃO DO POTENCIAL DE AÇÃO MUSCULAR ATRAVÉS DE TÉCNICAS DE APRENDIZAGEM 
DE MÁQUINAS APLICADAS À sEMG.}
Junior, J.L.F.S; Guimarães, I.J.A.; Lima, R.A.; Sousa, B.S; Brasil, L.M.; Marães, V.R.F.S.
Anais do V Congresso Brasileiro de Eletromiografia e Cinesiologia e X Simpósio de Engenharia Biomédica. 2017 \\

\textbf{PROCESSAMENTO E ANÁLISE DE INTERVALOS R-R PARA A VARIABILIDADE DA FREQUÊNCIA CARDÍACA.}
Sousa, B.S; Guimarães, I.J.A.; Junior, J.L.F.S; Souza G.S.L.; Brasil, L.M.; Marães, V.R.F.S.
Anais do V Congresso Brasileiro de Eletromiografia e Cinesiologia e X Simpósio de Engenharia Biomédica. 2017 \\

\textbf{Predicting knee angles from video - an initial experiment with Machine Learning.}
Guimarães, I.J.A.; Lopes, R.M.; Junior, J.L.F.S; Sousa, B.S; Marães, V.R.F.S.; Brasil, L.M.
XXVI Congresso Brasileiro de Engenharia Biomédica – CBEB. 2018 \\

\newpage
\section{Habilidades e conhecimentos:}
\hrule \vspace{0.1cm}

\subsection{Idiomas:}
\begin{center}
\begin{tabular}{c | c | c | c}
\hline
\textbf{Língua:} & Fala & Leitura & Escrita \\
\hline
\hline
\textbf{Inglês} &  Fluente & Fluente & Fluente  \\
\hline
\textbf{Espanhol} & Básico & Básico & Básico \\
\hline
\textbf{Francês} & Básico & Básico & Básico \\
\hline
\end{tabular}
\end{center}

\subsection{Habilidades diversas:}

\begin{multicols}{3} %number of columns
\begin{itemize}
    \item Python;
    \item Go/Golang;
    \item C;
    \item Shell script;
    \item Git;
    \item Gitflow;
    \item Docker;
    \item Kubernetes;
    \item PostgreSQL;
    \item C\#;
    \item \textit{Machine Learning};
    \item \LaTeX ;
    \item Matlab;
    \item Octave;
    \item Julia;
    \item Celery;
    \item RabbitMQ
    % \item Office (Word, Excel e PowerPoint);
    % \item LibreOffice;
    % \item Pages, Numbers e Keynote;
    % \item Windows, GNU/Linux e MacOS;
    \item Aquisição de sinais biomédicos;
    \item Instrumentação biomédica e eletrônica;
    \item Filtros analógicos e digitais;
    \item Arduino;
    \item Processamento digital de sinais;
    \item Criação de \textit{bots}.
\end{itemize}
\end{multicols}


\end{document}
