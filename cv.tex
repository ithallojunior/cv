\documentclass[11pt]{article}
\usepackage[utf8]{inputenc}
\usepackage[T1]{fontenc} %corrects accents
\usepackage[margin=1.0in]{geometry}
\usepackage[colorlinks=true, 
            urlcolor=blue,
            pdfauthor={Ithallo Junior Alves Guimaraes},
            pdftitle={CV de Ithallo},
            ]{hyperref}
\usepackage{scrextend} %addmargin
\usepackage{multicol} % multiline columns for itemize
\setlength{\parindent}{0cm}

\begin{document}

\begin{center}
\huge{Ithallo Junior Alves Guimarães}
\end{center}

\section{Informações pessoais:}
\hrule \vspace{0.1cm}
\begin{addmargin}{0.5cm}

\textbf{Endereço:}  QN 7 AE 7, Riacho Fundo 1, 71805-737, Brasília - DF. \\
\textbf{Telefone:}  +55 61 99951-7324 \\
\textbf{Email:}   \href{maito:ithallojunior@outlook.com}{ithallojunior@outlook.com} \\
\textbf{Data de nascimento:} 10/10/1995 \\
\textbf{Site:}  \url{https://ithallojunior.github.io}  \\
\textbf{GitHub:} \url{https://github.com/ithallojunior}

\end{addmargin}

\section{Apresentação}
\hrule \vspace{0.1cm}

\begin{addmargin}{0.5cm}

Mestrando em Engenharia Biomédica com formação em em Engenharia Eletrônica. 
Experiência em Python com mais de 4 anos, 
\textit{Machine Learning} com aproximadamente 3 anos, instrumentação biomédica (e 
eletrônica) com mais de 2 anos, 
aquisição de sinais biomédicos com mais de 2 anos, processamento digital de sinais
com mais de 2 anos, entre outras expertises. 
Autodidata, determinado e 
capaz de aprender rapidamente. Possui Inglês fluente e Francês e Espanhol básicos.

\end{addmargin}

\section{Formação:}
\hrule \vspace{0.1cm}

\textbf{2018 - presente}
\begin{addmargin}{0.5cm}
Mestrado em Engenharia Biomédica - Universidade de Brasília (UnB), Brasília - DF. \\
\end{addmargin}

\textbf{2012 - 2017}
\begin{addmargin}{0.5cm}
Graduação em Engenharia Eletrônica - Universidade de Brasília (UnB), Brasília - DF. \\
% Tema da monografia: Desenvolvimento tecnológico de um dispositivo de coleta
%de sinais sEMG aplicado à avaliação da doença de \textit{Parkinson}.\\
\end{addmargin}

\textbf{2014 - 2015} 
\begin{addmargin}{0.5cm}
Intercâmbio (Engenharia Elétrica) - Universidade de Tecnologia e Economia de Budapeste (BME), Budapeste, Hungria. 
\end{addmargin}

\section{Experiência:}
\hrule \vspace{0.1cm}

\textbf{março 2018 - presente}
\begin{addmargin}{0.5cm}
Bolsista - Coordenadoria de Aperfeiçoamento de Pessoal de Nível Superior (CAPES). \\
Bolsista pelo Programa de Pós-Graduação em Engenharia Biomédica
da Faculdade Gama da Universidade de Brasília (PPGEB FGA-UnB).\\
Linha de pesquisa: Análise e desenvolvimento de sistemas inteligentes e de saúde.\\
\end{addmargin}

\textbf{outubro 2018 - dezembro 2018}
\begin{addmargin}{0.5cm}
Professor voluntário - Faculdade Gama, Universidade de Brasília (FGA-UnB).\\
Professor de Prática de Eletrônica Digital 1. \\
\end{addmargin}

\textbf{março 2018 - julho 2018}
\begin{addmargin}{0.5cm}
Estágio em docência - Faculdade Gama, Universidade de Brasília (FGA-UnB).\\
Aulas ministradas na disciplina de Inteligência Artificial sobre Redes 
Neurais Artificiais e Algoritmos Genéticos. \\
\end{addmargin}

\textbf{setembro 2017 - fevereiro 2018}
\begin{addmargin}{0.5cm}
Desenvolvedor Python - LoopKey.\\
Desenvolvimento de sistemas de comunicação, automação e controle via Bluetooth
em sistemas Linux  embarcados utilizando-se Python. \\
\end{addmargin}


\textbf{janeiro 2016 - janeiro 2017}
\begin{addmargin}{0.5cm}
Bolsista - Fundação de Apoio a Pesquisa do Distrito Federal (FAP-DF).\\
Bolsista de Iniciação Científica no Laboratório de Informática em Saúde (LIS)  da Universidade de
Brasília (UnB) no projeto
de desenvolvimento tecnológico para adaptação de membros artificiais em amputados transfemorais,
trabalhando com desenvolvimento de circuitos eletrônicos, Python e aplicações de modelos de 
\textit{Machine Learning}.\\

\end{addmargin}

    \textbf{setembro 2014 - agosto 2015}
    \begin{addmargin}{0.5cm}
        Bolsista - Coordenadoria de Aperfeiçoamento de Pessoal de Nível Superior (CAPES). \\
        Bolsista de intercâmbio pelo programa Ciência sem Fronteiras na Universidadede Tecnologia e Economia de Budapeste (BME). 
\end{addmargin}

\section{Habilidades e conhecimentos:}
\hrule \vspace{0.1cm}

\subsection{Idiomas:}
\begin{center}
\begin{tabular}{c | c | c | c}
\hline
\textbf{Língua:} & Fala & Leitura & Escrita \\
\hline
\hline
\textbf{Inglês} &  Fluente & Fluente & Fluente  \\
\hline
\textbf{Espanhol} & Básico & Básico & Básico \\
\hline
\textbf{Francês} & Básico & Básico & Básico \\
\hline
\end{tabular}
\end{center}

\subsection{Habilidades diversas:}

\begin{multicols}{3} %number of columns
\begin{itemize}
    \item Python;
    \item C;
    \item Shell;
    \item Hadoop;
    \item C\#;
    \item Hive;
    \item SQL;
    \item MongoDB;
    \item \textit{Machine Learning};
    \item Spark;
    \item \LaTeX ;
    \item Git;
    \item AngularJS;
    \item Matlab;
    \item Octave;
    \item Julia;
    \item Office (Word, Excel e PowerPoint);
    \item LibreOffice;
    \item Pages, Numbers e Keynote;
    \item Windows, GNU/Linux e MacOS (OS X);
    \item Aquisição de sinais biomédicos;
    \item Instrumentação biomédica e eletrônica;
    \item Filtros analógicos e digitais;
    \item Arduíno;
    \item Processamento digital de sinais;
    \item Criação de \textit{bots}.
\end{itemize}
\end{multicols}


\end{document}
